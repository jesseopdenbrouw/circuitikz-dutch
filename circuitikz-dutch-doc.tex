% !TeX encoding = UTF-8
% !TeX TS-program = pdflatex
% !TeX spellcheck = en_GB

%%--------------------------------------------------------------------
%%--------------------------------------------------------------------
%%
%% circuitikz=dutch-doc.tex version 0.2
%%
%% This is the user's guide for the circuitikz-dutch package, 
%% by Jesse op den Brouw
%%
%% The package itself is in the file circuitikz-dutch.sty.


%% Copyright (c)2021, Jesse E. J. op den Brouw
%%
%% This work may be distributed and/or modified under the
%% conditions of the LaTeX Project Public License, either version 1.3
%% of this license or (at your option) any later version.
%% The latest version of this license is in
%%   http://www.latex-project.org/lppl.txt
%% and version 1.3 or later is part of all distributions of LaTeX 
%% version 2003/12/01 or later.
%%
%% This work consists of the files circuitikz-dutch.sty and
%% circuitikz-dutch-doc.tex 

%% This software is provided 'as is', without warranty of any kind,
%% either expressed or implied, including, but not limited to, the
%% implied warranties of merchantability and fitness for a
%% particular purpose.

%% Jesse op den Brouw
%% Department of Electrical Engineering
%% The Hague University of Applied Sciences
%% Rotterdamseweg 137, 2628 AL, Delft
%% Netherlands
%% J.E.J.opdenBrouw@hhs.nl

%% The newest version of this package should always be available
%% from https://github.com/jesseopdenbrouw/circuitikz-dutch


\documentclass[12pt]{article}

\usepackage[a4paper,left=3cm,right=3cm,top=3cm,bottom=3.5cm]{geometry}
\usepackage{parskip}
\usepackage{float}
\usepackage[font=footnotesize,format=plain,labelfont=bf,up,textfont=sl,up]{caption}
\usepackage[labelformat=simple,font=footnotesize,format=plain,labelfont=bf,textfont=sl]{subcaption}
\captionsetup[figure]{justification=centering,singlelinecheck=off,belowskip=-1ex}
\captionsetup[table]{justification=centering,singlelinecheck=off,skip=1ex}
\captionsetup[subtable]{justification=centering,singlelinecheck=off,skip=3pt}
\renewcommand\thesubfigure{(\alph{subfigure})}
\renewcommand\thesubtable{(\alph{subtable})}

%% Use the Charter font + math symbols
\usepackage[scaled=0.9]{nimbusmono}
\usepackage[bitstream-charter]{mathdesign}
%% Use microtype
\usepackage[stretch=10]{microtype}

\usepackage{xcolor}
\usepackage{listings}
\usepackage{textcomp}

\usepackage{url}
\lstset{ %
  language=[AlLaTeX]TeX,
  basicstyle=\small\ttfamily,
  numbers=left,
  numberstyle=\tiny\color{gray},
  stepnumber=1,                           
  numbersep=8pt,
  showspaces=false,
  showstringspaces=false,
  showtabs=false,
  frame=none,
  rulecolor=\color{gray},
  tabsize=4,
  captionpos=b,
  breaklines=true,
  breakatwhitespace=true,
  title=\lstname,
  upquote=true,
  aboveskip=\baselineskip,
  belowskip=-\baselineskip,
  escapeinside={(*}{*)},
}

\usepackage[colorlinks,linkcolor=blue]{hyperref}
\urlstyle{tt}

\usepackage{circuitikz-dutch}

\author{Jesse op den Brouw\thanks{Version 0.2, The Hague University of Applied Sciences, \href{mailto:J.E.J.opdenBrouw@hhs.nl}{\texttt{J.E.J.opdenBrouw@hhs.nl}}}}
\title{\Huge \texttt{circuitikz-dutch} \\[1.5ex]\Large Drawing Electric Circuits in Dutch Textbooks}
\date{\today}

\def\Circuitikz{CircuiTi\emph{k}Z}

\begin{document}
\maketitle

This package sets up \Circuitikz{} to draw electric circuits with the
conventions used in Dutch textbooks. After loading the \texttt{circuitikz}
package, a number of patches will be applied to \Circuitikz{} commands.


\subsubsection*{License and warranty}
This work may be distributed and/or modified under the
conditions of the \LaTeX{} Project Public
License, either version~1.3 of this
license or (at your option) any later version. The latest version
of this license is in \url{http://www.latex-project.org/lppl.txt}
and version 1.3 or later is part of all distributions of \LaTeX{}
version 2003/12/01 or later.

This work has the LPPL maintenance status ``author-maintained''.

This work consists of the files \texttt{circuitikz-dutch.sty}
and \texttt{circuitikz-dutch-doc.tex}

This software is provided `as is', without warranty of any kind,
either expressed or implied, including, but not limited to, the
implied warranties of merchantability and fitness for a
particular purpose.

\subsubsection*{Package loading}
Load the package by:

\begin{lstlisting}
\usepackage{circuitikz-dutch}
\end{lstlisting}

The package has no options. \Circuitikz{} will be loaded by this package. Please
\textbf{do not} (re)load \Circuitikz{} after this package.

\subsubsection*{The symbols}
 The symbols for voltage source, current source, resistor, capacitor and inductor are shown in Figure~\ref{symbols1}.


\begin{figure}[!ht]
\centering
\begin{circuitikz}
\draw (0,0) to[V=$U$] ++(0,2);
\draw (3,0) to[I,label=$I$] ++(0,2);
\draw (6,0) to[R=$R$] ++(0,2);
\draw (9,0) to[C=$C$] ++(0,2);
\draw (12,0) to[L=$L$] ++(0,2);
\end{circuitikz}
\caption{Dutch symbols for voltage source, current source, resistor, capacitor and inductor.}
\label{symbols1}
\end{figure}

The \Circuitikz{} code to produce these symbols is shown below:

\begin{lstlisting}
\begin{circuitikz}
\draw (0,0) to[V=$U$] ++(0,2);        % independent voltage source
\draw (2,0) to[I,label=$I$] ++(0,2);  % independent current source
\draw (4,0) to[R=$R$] ++(0,2);        % resistor
\draw (6,0) to[C=$C$] ++(0,2);        % capacitor
\draw (8,0) to[L=$L$] ++(0,2);        % inductor
\end{circuitikz}
\end{lstlisting}

\textbf{Note:} due to the present current drawing strategy, it's mandatory that you use a label when specifying the source's current. See Figure~\ref{symbols2} below for the differences.

\begin{figure}[!ht]
\centering
\begin{circuitikz}
\draw (3,0) to[I,label=$I$] ++(0,2);
\draw (6,0) to[I=$I$] ++(0,2);
\end{circuitikz}
\caption{Differences in drawing current sources.}
\label{symbols2}
\end{figure}

The \Circuitikz{} code to produce these symbols is shown below:

\begin{lstlisting}
\begin{circuitikz}
\draw (3,0) to[I,label=$I$] ++(0,2);
\draw (6,0) to[I=$I$] ++(0,2);
\end{circuitikz}
\end{lstlisting}

In Figure~\ref{symbols3}, the dependent voltage and current sources are shown:

\begin{figure}[!ht]
\centering
\begin{circuitikz}
\draw (0,0) to[cvsource=$\alpha U$] ++(0,2);
\draw (3,0) to[cvsource=$\beta I$] ++(0,2);
\draw (6,0) to[cisource,label=$\gamma U$] ++(0,2);
\draw (9,0) to[cisource,label=$\delta I$] ++(0,2);
\end{circuitikz}
\caption{Symbols for dependant sources.}
\label{symbols3}
\end{figure}

The code is shown below:

\begin{lstlisting}
\begin{circuitikz}
\draw (0,0) to[cvsource=$\alpha U$] ++(0,2);
\draw (3,0) to[cvsource=$\beta I$] ++(0,2);
\draw (6,0) to[cisource,label=$\gamma U$] ++(0,2);
\draw (9,0) to[cisource,label=$\delta I$] ++(0,2);
\end{circuitikz}
\end{lstlisting}

Sinusiodal sources are shown below:

\begin{figure}[H]
\centering
\begin{circuitikz}
\draw (0,0) to[sV=$\hat{u}\,\mathrm{e}^{\,\mathrm{j}\varphi}$] ++(0,2);
\draw (3,0) to[sI,label=$\hat{\imath}\,\mathrm{e}^{\,\mathrm{j}\varphi}$] ++(0,2);
\end{circuitikz}
\caption{Symbols for sinusiodal sources.}
\label{symbols5}
\end{figure}

Code to produce these symbols:

\begin{lstlisting}
\begin{circuitikz}
\draw (0,0) to[sV=$\hat{u}\,\mathrm{e}^{\,\mathrm{j}\varphi}$] ++(0,2);
\draw (3,0) to[sI,label=$\hat{\imath}\,\mathrm{e}^{\,\mathrm{j}\varphi}$] ++(0,2);
\end{circuitikz}
\end{lstlisting}

Batteries are shown in Figure~\ref{symbols7}.

\begin{figure}[H]
\centering
\begin{circuitikz}
\draw (3,0) to[battery=$U$,invert] ++(0,2);
\draw (6,0) to[battery1=$U$,invert] ++(0,2);
\draw (9,0) to[battery2=$U$,invert] ++(0,2);
\end{circuitikz}
\caption{Symbols for batteries.}
\label{symbols7}
\end{figure}

Code to produce these symbols:

\begin{lstlisting}
\begin{circuitikz}
\draw (3,0) to[battery=$U$,invert] ++(0,2);
\draw (6,0) to[battery1=$U$,invert] ++(0,2);
\draw (9,0) to[battery2=$U$,invert] ++(0,2);
\end{circuitikz}
\end{lstlisting}

Voltages across components are displayed with `+' and `$-$', as shown in Figure~\ref{symbols6}.

\begin{figure}[!ht]
\centering
\begin{circuitikz}
\draw (0,0) to[R=$R$, v=$U_R$] ++(0,2);
\draw (3,0) to[C=$C$, v=$U_C$] ++(0,2);
\draw (6,0) to[L=$L$, v=$U_L$] ++(0,2);
\draw (9,0) to[I,label=$I$, v>=$U_I$] ++(0,2);
\end{circuitikz}
\caption{Symbols with voltages.}
\label{symbols6}
\end{figure}

The code to produce the symbols is shown below:

\begin{lstlisting}
\begin{circuitikz}
\draw (0,0) to[R=$R$, v=$U_R$] ++(0,2);
\draw (3,0) to[C=$C$, v=$U_C$] ++(0,2);
\draw (6,0) to[L=$L$, v=$U_L$] ++(0,2);
\draw (9,0) to[I,label=$I$, v>=$U_I$] ++(0,2);
\end{circuitikz}
\end{lstlisting}

\subsubsection*{Displaying values and units}

You can use the \href{https://ctan.org/pkg/siunitx}{\texttt{siunitx}} package to put values and units to the symbols:

\begin{figure}[!ht]
\centering
\begin{circuitikz}
\draw (0,0) to[V=\SI{12}{\volt}] ++(0,2);
\draw (3,0) to[I,label=\SI{10}{\milli\ampere}] ++(0,2);
\draw (6,0) to[R=\SI{22}{\kilo\ohm}] ++(0,2);
\draw (9,0) to[C=\SI{47}{\nano\farad}] ++(0,2);
\draw (12,0) to[L=\SI{15}{\micro\henry}] ++(0,2);
\end{circuitikz}
\caption{Symbols with values and units.}
\label{symbols4}
\end{figure}

The code to produce these symbols:

\begin{lstlisting}
\begin{circuitikz}
\draw (0,0) to[V=\SI{12}{\volt}] ++(0,2);
\draw (3,0) to[I,label=\SI{10}{\milli\ampere}] ++(0,2);
\draw (6,0) to[R=\SI{22}{\kilo\ohm}] ++(0,2);
\draw (9,0) to[C=\SI{47}{\nano\farad}] ++(0,2);
\draw (12,0) to[L=\SI{15}{\micro\henry}] ++(0,2);
\end{circuitikz}
\end{lstlisting}

\subsubsection*{Rotating current sources}
Using the \texttt{mirror} and \texttt{invert} options, together with using \texttt{l\_}, it is possible to rotate current sources.

\begin{figure}[!ht]
\centering
\begin{circuitikz}
\draw (3,0) to[I,l=$I$] ++(0,2);
\draw (6,0) to[I,l_=$I$,mirror] ++(0,2);
\draw (9,0) to[I,l=$I$,invert] ++(0,2);
\draw (12,0) to[I,l_=$I$,mirror,invert] ++(0,2);
\end{circuitikz}
\caption{Different orientations of current sources.}
\label{symbols8}
\end{figure}

The code to produce these symbols:

\begin{lstlisting}
\begin{circuitikz}
\draw (3,0) to[I,l=$I$] ++(0,2);
\draw (6,0) to[I,l_=$I$,mirror] ++(0,2);
\draw (9,0) to[I,l=$I$,invert] ++(0,2);
\draw (12,0) to[I,l_=$I$,mirror,invert] ++(0,2);
\end{circuitikz}
\end{lstlisting}

\subsubsection*{Changes to the \texttt{circuitikz} package}
The \texttt{circuitikz-dutch} package changes the following:

\begin{itemize}
\item Voltage sources have `+' and `$-$' glyphs left or right;
\item Current sources have an arrow left or right;
\item Resistors are of European type;
\item Inductors are of American type;
\item Voltages across components have `+' and `$-$' glyphs left or right;
\item Line width is set to 1 pt for all Ti\emph{k}Z pictures;
\item Relative thickness of bipoles, tripoles and quadpoles are set to 1;
\item The proportions of the components are set to better proportions;
\item The package \texttt{siunitx} is loaded;
\end{itemize}


\subsubsection*{Example network}

In Figure~\ref{fig:gelmodelopampinverter} you see a network with a simplified model of an opamp.

\begin{figure}[H]
\centering
\begin{tikzpicture}[scale=1.25]
\draw (0,0) to[V, v=$U$] ++(0,2) node[above] {$U_1$}
            to[R, R=$R_1$, -*] ++(2,0) node (2) {} node[below] {$U_x$}
 (2.center) to[short] ++(1,0)
            to[open] ++(0,-2)
            to[R, a=$R_{in}$, v^<=$U_{in}$,*-] ++(0,2) 
 (2.center) to[short] ++(0,1)
            to[R=$R_2$] ++(5.5,0)
            to[short,-*] ++(0,-1) node (3) {}
			to[short,-o] ++(0.5,0) node[right] {$U_{2}$}
 (3.center) to[R=$R_{out}$] ++(-2,0)  node [above] {$U_y$}
            to[cV, v_<=$aU_{in}$,-*] ++(0,-2)
			to[open] ++(2.5,0)
            to[short, o-.] (0,0)
;
\end{tikzpicture}
\caption{Network with a simplified model of an opamp.}
\label{fig:gelmodelopampinverter}
\end{figure}

The code to produce this circuit is show below:

\begin{lstlisting}
\begin{tikzpicture}[scale=1.25]
\draw (0,0) to[V, v=$U$] ++(0,2) node[above] {$U_1$}
            to[R, R=$R_1$, -*] ++(2,0) node (2) {} node[below] {$U_x$}
 (2.center) to[short] ++(1,0)
            to[open] ++(0,-2)
            to[R, a=$R_{in}$, v^<=$U_{in}$,*-] ++(0,2) 
 (2.center) to[short] ++(0,1)
            to[R=$R_2$] ++(5.5,0)
            to[short,-*] ++(0,-1) node (3) {}
			to[short,-o] ++(0.5,0) node[right] {$U_{2}$}
 (3.center) to[R=$R_{out}$] ++(-2,0)  node [above] {$U_y$}
            to[cV, v_<=$aU_{in}$,-*] ++(0,-2)
			to[open] ++(2.5,0)
            to[short, o-.] (0,0)
;
\end{tikzpicture}
\end{lstlisting}
\subsubsection*{Changelog}

\begin{table}[!ht]
\begin{tabular}{lll}
\textbf{Version} & \textbf{Date} & \textbf{What} \\
v0.1    & 2020/03/08 & First release\\
v0.2    & 2021/06/12 & Compliant with Circuitikz 1.3.7
\end{tabular}
\end{table}



\end{document}